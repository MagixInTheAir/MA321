\chapter{Introduction}
La résolution d'un système d'équations linéaires $Ax = b$ est probablement le problème le plus basique de l'algèbre linéaire. Il y a 4000 ans, les Babyloniens étaient déjà capables de résoudre des systèmes d'équations linéaires de taille $2\times 2$. Le texte chinois \underline{Les Neufs Chapitres sur l'art mathématique} rédigé au II\ieme{} siècle avant J.-C, parlait déjà de l'utilisation de tableaux de nombre pour résoudre des sytèmes d'équations.\\

 Cependant, ce n'est qu'en 1750 que Cramer établit une méthode de résolution de systèmes $n\times n$. Mais la règle de Cramer devient difficilement fastidieuse dès lors que le système dépasse tois équations. Au début du XIX\ieme{} siècle, Gauss propose sa méthode du pivot, encore largement utilisée aujourd'hui. \\

Enfin, en 1848, Sylvester introduit le terme "matrice". Quelques années plus tard est publié le théorème de Cayley-Hamilton. Puis les avancées mathématiques ralentissent dans ce domaine, jusqu'à la Seconde Guerre mondiale et le développement de l'informatique.\\

Avec l'arrivée des ordinateurs, les méthodes de résolutions de systèmes d'équations linéaires deviennent applicables à des systèmes de grande taille. L'augmentation des puissances de calculs permettent de développer de nouvelles méthodes de résolution, pensées directement pour des machines.\\

Deux grandes familles de méthodes sont développées en parallèle : les méthodes directes et les méthodes itératives. Ce mémoire est consacré aux méthodes itératives ainsi qu'à leur optimisation, du point de vue mathématique puis numérique.
\chapter{Conclusion \& ouverture}

Énormément de problèmes peuvent aujourd'hui se caractériser où s'approximer par un système linéaire qu'il faut résoudre. C'est dans cette optique que nous avons étudié de nouveaux moyens de résoudre, à travers les méthodes itératives, ces systèmes. En partant de la base de ces méthodes, et en complexifiant à chaque étape le procédé de résolution nous sommes parvenus à améliorer nos algorithmes.\\

La première étape fut de partir des algorithmes de Jacobi et Gauss-Seidel. En introduisant une décomposition particulière sous la forme $M - N$, nous avons pu paramétrer la décomposition par un nombre, nous permettant de redécouvrir la méthode de Richardson. Celle-ci permet de choisir pour chaque matrice un nombre $\alpha$ optimal qui accroît la vitesse de convergence. Cette nouvelle méthode sera ensuite améliorée par le procédé de relaxation et l'algorithme SOR qui donne d'encore meilleurs résultats. L'amélioration de ce dernier, par la méthode SSOR a également été présentée. \\

Cependant, cela n'était pas suffisant car un problème n'avait pas encore de solution. En effet, il fallait nous assurer que nos algorithmes étaient robustes mais surtout que nous pouvions assurer leur convergence en un nombre fini d'itérations relativement petit et connu à l'avance si possible. Dans ce but, nous avons introduit les espaces de Krylov. Ils nous ont permis d'obtenir un cadre théorique et des méthodes surpassant tous les autres procédés itératifs par leur précision, leur rapidité, leur robustesse et leur application à n'importe quel cas. Ce cadre théorique a été ensuite implémenté par l'algorithme GMRES que nous avons codé.\\

La dernière partie de notre travail a porté sur l'optimisation numérique de ces méthodes afin de faire converger les algorithmes sur un ordinateur. Notre travail était principalement de faire quelques tests de benchmark sur les différents procédés.\\ 

Néanmoins, le travail effectué dans ce rapport n'est qu'un premier pas. En effet, nous sommes capable de résoudre très convenablement tous les problèmes linéaires qui peuvent se présenter à nous. Cependant, il ne faut pas oublier qu'en général les problèmes ne sont pas linéaires. Ainsi, il faut maintenant trouver un moyen de linéariser les problèmes. De plus, nos méthodes s'appliquent lorsque le système linéaire possède une unique solution mais certains systèmes n'ont pas de solutions ou une infinité. Il faudra donc construire de nouvelles méthodes pour ces systèmes.
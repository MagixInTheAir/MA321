\chapter{Conclusion \& ouverture}

Énormément de problèmes peuvent aujourd'hui se caractériser où s'approximer par un système linéaire qu'il faut résoudre. C'est dans cette optique que nous avons étudié de nouveaux moyens de résoudre, à travers les méthodes itératives, ces systèmes. En partant de la base de ces méthodes, et en complexifiant à chaque étape le procédé de résolution nous parvenu à améliorer nos algorithme.\\

La première étape fut de partir du simple algorithme de Jacobi et de Gauss-Seidel. En introduisant une décomposition particulière sous la forme M - N, nous avons pu paramétrer la décomposition par un nombre qui nous a permis de redécouvrir la méthode de Richardson qui permet de choisir pour chaque matrice un nombre $\alpha$ optimal qui accroître la vitesse de convergence. Cette nouvelle méthode sera ensuite améliorer par le procédé de relaxation et l'algorithme SOR qui donne d'encore meilleurs résultats. L'amélioration de ce dernier, par la méthode SSOR a aussi été présentée. \\

Cependant, cela n'était pas suffisant car un problème n'avait pas encore de solution. En effet, il nous fallait s'assurer que nos algorithmes soient robustes mais surtout que l'on puisse assurer la convergence en un nombre fini d'itération relativement petit et connu à l'avance si possible. C'est pour cela que l'on a introduit les espaces de Krylov. Ils nous ont permis d'obtenir un cadre théorique et des méthodes mettant à l'amande tous les autres procédés itératifs par sa précision, sa rapidité, sa robustesse et son application à n'importe quel cas. Ce cadre théorique a été ensuite implémenté par l'algorithme GMRES que nous avons codé. 

La dernière partie de notre travail a porté sur l'optimisation numérique de ces méthodes afin de faire converger les algorithmes sur un ordinateur. Notre travail était principalement de faire quelques tests de benchmark sur les différents procédé.\\ 

Néanmoins, le travail effectué dans ce rapport n'est qu'un premier pas. En effet, nous sommes capable de résoudre très convenablement tous les problèmes linéaires qui peuvent se présenter à nous. Cependant, il ne faut pas oublier qu'en générale les problèmes ne sont pas linéaire. Ainsi il faut maintenant trouver un moyen de linéariser les problèmes. De plus, nos méthodes s'appliquent lorsque le système linéaire possède une unique solution mais certains systèmes n'ont pas de solutions ou une infinité. Il faudra donc construire de nouvelles méthodes pour ces systèmes.